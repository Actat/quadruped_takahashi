\documentclass[twocolumn]{jlreq}
\usepackage[haranoaji]{luatexja-preset}

\usepackage{amsmath, amssymb}
\usepackage{bm}
\usepackage{siunitx}

\usepackage{graphicx}
\usepackage{svg}
\usepackage{wrapfig}

\begin{document}
  \title{document}
  \author{Actat}
  \date{}
  \maketitle

  \section{ロボット製作の動機}

  このロボットの製作に取り組むのは,歩行ロボット製作を経験するためである.
  最終的には二足歩行ロボットの製作を目指しているが,6自由度の脚の設計は難易度が高いと判断し,
  今回は3自由度の脚を4つ備えた四足歩行ロボットを製作することにした.

  このロボットの製作では以下の内容を経験することを期待している.
  \begin{itemize}
    \item 自宅でのロボット製作
    \item サーボモータを用いた機構の設計
    \item 最低限の電気回路の取扱い
    \item ROSを用いたプログラミング
    \item 歩行ロボットの制御
  \end{itemize}

  \section{機械設計・製作}

  設計・製作したロボットを図\ref{fig:robot}に示す.
  脚は前後左右に鏡像になっている同一の機構とした.
  サーボモータは研究室で使用しているモータと合わせて近藤科学株式会社のB3M-SC-1170-Aを用いた.
  CIT Brainsが公開しているB3Mモータを用いた二足歩行ロボットを参考に,関節間の距離は\SI{100}{mm}にした.

  胴体中央下部に株式会社アールティのUSB出力9軸IMUセンサモジュールを搭載した.
  センサの位置をロボットのベースリンクと一致させて計算を簡単にする意図がある.
  胴体上部前方に取り付けられた板にサーボモータとPCの間の通信のための基板を固定した.

  \begin{figure}[tb]
    \centering
    \input{robot.pdf_tex}
    \caption{
      設計・製作したロボット
    }
    \label{fig:robot}
  \end{figure}

  \section{電装}

  ロボット全体で12個のモータが使われている.
  モータとパソコンはRS485USB/シリアル変換アダプターを介して通信する.
  直列に接続できるモータの数は限られているので
  XHコネクター用ハブ typeAによって各脚ごとの系統に分けて接続した.
  配線の様子を図\ref{fig:wiring}に示す.

  XHコネクター用ハブには5つのコネクタがあり,すべて並列に接続される.
  このハブはロボット前方に横向きに取り付けられており,
  中央のコネクタをRS485USB/シリアル変換アダプターに接続し,
  左右のコネクタをそれぞれ脚のモータと接続した.
  脚の3つのモータはHip flexion/extension, Hip ab/adduction, kneeの順に接続された.
  Hipのモータは機械的な接続とは逆の順序である.

  \begin{figure}[tb]
    \centering
    \input{wiring.pdf_tex}
    \caption{
      配線の様子
    }
    \label{fig:wiring}
  \end{figure}

  サーボモータは配線に先立ってIDの書き込みを行った.
  書き込んだIDは表\ref{table:servo_id}に示すとおりである.

  \begin{table}[tb]
    \caption{IDと関節の対応}
    \label{table:servo_id}
    \centering
    \begin{tabular}{rcc}
      \hline
      id & 脚 & 関節 \\
      \hline
      0 & 左前 & Hip ab/adduction \\
      1 & 左前 & Hip flexion/extension \\
      2 & 左前 & knee \\
      3 & 右前 & Hip ab/adduction \\
      4 & 右前 & Hip flexion/extension \\
      5 & 右前 & knee \\
      6 & 左後 & Hip ab/adduction \\
      7 & 左後 & Hip flexion/extension \\
      8 & 左後 & knee \\
      9 & 右後 & Hip ab/adduction \\
      10 & 右後 & Hip flexion/extension \\
      11 & 右後 & knee \\
      \hline
    \end{tabular}
  \end{table}

\end{document}
